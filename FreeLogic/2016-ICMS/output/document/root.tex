
%==================================================
% ICMS LaTeX Template
%==================================================

%===== DO NOT MODIFY ==============================
\documentclass[runningheads,a4paper]{llncs}
\usepackage{amssymb}
\setcounter{tocdepth}{3}
%\usepackage{graphicx}
\usepackage{url}
\newcommand{\keywords}[1]{\par\addvspace\baselineskip
\noindent\keywordname\enspace\ignorespaces#1}


\usepackage{latexsym,amsmath}
\usepackage{tikz}
\usepackage{isabelle,isabellesym}

\usepackage{pdfsetup}

% urls in roman style, theory text in math-similar italics
\urlstyle{rm}
\isabellestyle{it}



\begin{document}

\mainmatter
%================================================


%==== FILL IN ====================================
\title{Automating Free Logic in Isabelle/HOL}  % Full title
\titlerunning{Short title} % Shor ttitle
\author{Christoph Benzm\"uller\inst{1} \and  Dana Scott\inst{2}}
\authorrunning{Benzm\"uller-Scott}
\institute{
Freie Universit\"at Berlin, Germany \& Visiting Scholar at Stanford University, USA\\
\email{c.benzmueller@fu-berlin.de},\\ 
\texttt{http://www.christoph-benzmueller.de}
\and
Visiting Scholar at University of Califormia, Berkeley,  USA\\
\email{dana.scott@cs.cmu.edu },\\ 
\texttt{http://www.cs.cmu.edu/~scott/}
}
\maketitle

\begin{abstract}
We present an interactive and automated theorem prover for free higher-order
logic. Our implementation on top of the Isabelle/HOL framework utilizes a semantic embedding of 
free logic in classical higher-order logic.
The capabilities of our tool are demonstrated with first experiments
in category theory.
\keywords{Free logic, interactive and automated theorem proving, model
  finding, application to category theory}
\end{abstract}


% %--- Remove this after reading it.
% \framebox{
% \begin{minipage}{5in}
% The main body should describe challenge, achievements and progress in
% mathematical software research, addressing issues such as
% functionality, underlying theories, design, development and applications. 
% \begin{itemize}
%  \item The whole paper (including the title page and the references) must be
%        \begin{itemize}
%        \item[] {\bf at least 4 pages} 
%        \item[] {\bf at most 8 pages}.
%        \end{itemize}
%  \item For a new software, some comparison with  existing software (if exists) will be appropriate. 
%  \item Carefully divide the main body into several meaningful sections. \\
%        For example, you could have sections such as the following. \\
%        They are {\bf \em not} required.  These are only given as an illustration.
% \end{itemize}
% \end{minipage}
% }

% %------------------------------------------------------------
% \section{Introduction}

% %------------------------------------------------------------
% \section{Functionality??}
% \begin{itemize}
% \item Describe the functionality of the software.
% \item Address  wide audience (not just experts on the particular area).
% \item Use carefully chosen  simple(toy) examples, screen shots, etc.
% \end{itemize}
  
% %------------------------------------------------------------
% \section{Application??}
% \begin{itemize}
% \item Show important/non-trivial applications of the software.
% \item Address wide audience (not just experts on the particular area).
% \item Use carefully chosen  non-trivial  examples, screen shots, etc.
% \end{itemize}

% %------------------------------------------------------------
% \section{Underlying theory??}
% \begin{itemize}
% \item Explain briefly/informally the main theories/algorithms underlying the software.
% \item Address wide audience (not just experts on the particular area).
% \item If helps, use carefully chosen  simple  examples, diagrams, etc.
% \end{itemize}

% %------------------------------------------------------------
% \section{Technical contribution??}
% \begin{itemize}
% \item Discuss main technical challenges that had to be overcome, 
%       while developing the software.
% \item Address experts on the particular area.
% \item If helps, use carefully chosen  simple  examples, diagrams, etc.
% \end{itemize}

% %------------------------------------------------------------
% \begin{thebibliography}{4}
% \bibitem{...} .... 
% \end{thebibliography}
% \end{document}




% for uniform font size
%\renewcommand{\isastyle}{\isastyleminor}





%\maketitle

%\tableofcontents

% sane default for proof documents
\parindent 0pt\parskip 0.5ex

% generated text of all theories
\input{FreeFOL.tex}

%
\begin{isabellebody}%
\setisabellecontext{Freyd}%
%
\isadelimtheory
%
\endisadelimtheory
%
\isatagtheory
\isacommand{theory}\isamarkupfalse%
\ Freyd\ \isakeyword{imports}\ FreeFOL\ \isanewline
\isakeyword{begin}%
\endisatagtheory
{\isafoldtheory}%
%
\isadelimtheory
\isanewline
%
\endisadelimtheory
\isacommand{type{\isacharunderscore}synonym}\isamarkupfalse%
\ e\ {\isacharequal}\ i\ \ \ %
\isamarkupcmt{raw type of morphisms%
}
\ \isanewline
\isanewline
\isacommand{abbreviation}\isamarkupfalse%
\ OrdinaryEquality\ {\isacharcolon}{\isacharcolon}\ {\isachardoublequoteopen}e{\isasymRightarrow}e{\isasymRightarrow}bool{\isachardoublequoteclose}\ {\isacharparenleft}\isakeyword{infix}{\isachardoublequoteopen}{\isasymapprox}{\isachardoublequoteclose}{\isadigit{6}}{\isadigit{0}}{\isacharparenright}\ \isanewline
\ \isakeyword{where}\ {\isachardoublequoteopen}x\ {\isasymapprox}\ y\ {\isasymequiv}\ {\isacharparenleft}{\isacharparenleft}E\ x{\isacharparenright}\ \isactrlbold {\isasymleftrightarrow}\ {\isacharparenleft}E\ y{\isacharparenright}{\isacharparenright}\ \isactrlbold {\isasymand}\ x\ \isactrlbold {\isacharequal}\ y{\isachardoublequoteclose}\ \ \isanewline
\isanewline
\isacommand{consts}\isamarkupfalse%
\ source\ {\isacharcolon}{\isacharcolon}\ {\isachardoublequoteopen}e{\isasymRightarrow}e{\isachardoublequoteclose}\ {\isacharparenleft}{\isachardoublequoteopen}{\isasymbox}{\isacharunderscore}{\isachardoublequoteclose}\ {\isacharbrackleft}{\isadigit{1}}{\isadigit{0}}{\isadigit{8}}{\isacharbrackright}{\isadigit{1}}{\isadigit{0}}{\isadigit{9}}{\isacharparenright}\ \isanewline
\ \ \ \ \ \ \ target\ {\isacharcolon}{\isacharcolon}\ {\isachardoublequoteopen}e{\isasymRightarrow}e{\isachardoublequoteclose}\ {\isacharparenleft}{\isachardoublequoteopen}{\isacharunderscore}{\isasymbox}{\isachardoublequoteclose}\ {\isacharbrackleft}{\isadigit{1}}{\isadigit{1}}{\isadigit{0}}{\isacharbrackright}{\isadigit{1}}{\isadigit{1}}{\isadigit{1}}{\isacharparenright}\ \isanewline
\ \ \ \ \ \ \ composition\ {\isacharcolon}{\isacharcolon}\ {\isachardoublequoteopen}e{\isasymRightarrow}e{\isasymRightarrow}e{\isachardoublequoteclose}\ {\isacharparenleft}\isakeyword{infix}\ {\isachardoublequoteopen}{\isasymcdot}{\isachardoublequoteclose}\ {\isadigit{1}}{\isadigit{1}}{\isadigit{0}}{\isacharparenright}\isanewline
\isanewline
\isacommand{axiomatization}\isamarkupfalse%
\ FreydsAxioms\ \isakeyword{where}\ \ \ \ \ \ \ \ \ \ \ \ \ \ \ \isanewline
\ A{\isadigit{1}}{\isacharcolon}\ \ {\isachardoublequoteopen}{\isacharparenleft}E\ x{\isasymcdot}y{\isacharparenright}\ \isactrlbold {\isasymleftrightarrow}\ {\isacharparenleft}{\isacharparenleft}x{\isasymbox}{\isacharparenright}\ {\isasymapprox}\ {\isacharparenleft}{\isasymbox}y{\isacharparenright}{\isacharparenright}{\isachardoublequoteclose}\ \isakeyword{and}\isanewline
\isanewline
\ A{\isadigit{2}}b{\isacharcolon}\ {\isachardoublequoteopen}{\isasymbox}{\isacharparenleft}x{\isasymbox}{\isacharparenright}\ {\isasymapprox}\ {\isasymbox}x{\isachardoublequoteclose}\ \isakeyword{and}\isanewline
\ A{\isadigit{3}}a{\isacharcolon}\ {\isachardoublequoteopen}{\isacharparenleft}{\isasymbox}x{\isacharparenright}{\isasymcdot}x\ {\isasymapprox}\ x{\isachardoublequoteclose}\ \isakeyword{and}\isanewline
\ A{\isadigit{3}}b{\isacharcolon}\ {\isachardoublequoteopen}x{\isasymcdot}{\isacharparenleft}x{\isasymbox}{\isacharparenright}\ {\isasymapprox}\ x{\isachardoublequoteclose}\ \isakeyword{and}\isanewline
\ A{\isadigit{4}}a{\isacharcolon}\ {\isachardoublequoteopen}{\isasymbox}{\isacharparenleft}x{\isasymcdot}y{\isacharparenright}\ {\isasymapprox}\ {\isasymbox}{\isacharparenleft}x{\isasymcdot}{\isacharparenleft}{\isasymbox}y{\isacharparenright}{\isacharparenright}{\isachardoublequoteclose}\ \isakeyword{and}\isanewline
\ A{\isadigit{4}}b{\isacharcolon}\ {\isachardoublequoteopen}{\isacharparenleft}x{\isasymcdot}y{\isacharparenright}{\isasymbox}\ {\isasymapprox}\ {\isacharparenleft}{\isacharparenleft}x{\isasymbox}{\isacharparenright}{\isasymcdot}y{\isacharparenright}{\isasymbox}{\isachardoublequoteclose}\ \isakeyword{and}\isanewline
\ A{\isadigit{5}}{\isacharcolon}\ \ {\isachardoublequoteopen}x{\isasymcdot}{\isacharparenleft}y{\isasymcdot}z{\isacharparenright}\ {\isasymapprox}\ {\isacharparenleft}x{\isasymcdot}y{\isacharparenright}{\isasymcdot}z{\isachardoublequoteclose}\isanewline
\isanewline
\isanewline
\isanewline
\isacommand{lemma}\isamarkupfalse%
\ A{\isadigit{2}}a{\isacharcolon}\ {\isachardoublequoteopen}{\isacharparenleft}{\isasymbox}x{\isacharparenright}{\isasymbox}\ {\isasymapprox}\ {\isasymbox}x{\isachardoublequoteclose}\ \isanewline
%
\isadelimproof
\ %
\endisadelimproof
%
\isatagproof
\isacommand{proof}\isamarkupfalse%
\ {\isacharminus}\ \ \ \ \ \ \ \ \ \ \ \ \ \ \ \ \ \ \ \ \isanewline
\ \isacommand{have}\isamarkupfalse%
\ \ L{\isadigit{1}}{\isacharcolon}\ \ {\isachardoublequoteopen}{\isasymforall}x{\isachardot}\ {\isacharparenleft}{\isasymbox}{\isasymbox}x{\isacharparenright}{\isasymcdot}{\isacharparenleft}{\isacharparenleft}{\isasymbox}x{\isacharparenright}{\isasymcdot}x{\isacharparenright}\ {\isasymapprox}\ {\isacharparenleft}{\isacharparenleft}{\isasymbox}{\isasymbox}x{\isacharparenright}{\isasymcdot}{\isacharparenleft}{\isasymbox}x{\isacharparenright}{\isacharparenright}{\isasymcdot}x{\isachardoublequoteclose}\ \isacommand{using}\isamarkupfalse%
\ A{\isadigit{5}}\ \isacommand{by}\isamarkupfalse%
\ metis\isanewline
\ \isacommand{hence}\isamarkupfalse%
\ L{\isadigit{2}}{\isacharcolon}\ \ {\isachardoublequoteopen}{\isasymforall}x{\isachardot}\ {\isacharparenleft}{\isasymbox}{\isasymbox}x{\isacharparenright}{\isasymcdot}x\ {\isasymapprox}\ {\isacharparenleft}{\isacharparenleft}{\isasymbox}{\isasymbox}x{\isacharparenright}{\isasymcdot}{\isacharparenleft}{\isasymbox}x{\isacharparenright}{\isacharparenright}{\isasymcdot}x{\isachardoublequoteclose}\ \ \ \ \ \ \ \ \isacommand{using}\isamarkupfalse%
\ A{\isadigit{3}}a\ \isacommand{by}\isamarkupfalse%
\ metis\isanewline
\ \isacommand{hence}\isamarkupfalse%
\ L{\isadigit{3}}{\isacharcolon}\ \ {\isachardoublequoteopen}{\isasymforall}x{\isachardot}\ {\isacharparenleft}{\isasymbox}{\isasymbox}x{\isacharparenright}{\isasymcdot}x\ {\isasymapprox}\ {\isacharparenleft}{\isasymbox}x{\isacharparenright}{\isasymcdot}x{\isachardoublequoteclose}\ \ \ \ \ \ \ \ \ \ \ \ \ \ \ \ \isacommand{using}\isamarkupfalse%
\ A{\isadigit{3}}a\ \isacommand{by}\isamarkupfalse%
\ metis\isanewline
\ \isacommand{hence}\isamarkupfalse%
\ L{\isadigit{4}}{\isacharcolon}\ \ {\isachardoublequoteopen}{\isasymforall}x{\isachardot}\ {\isacharparenleft}{\isasymbox}{\isasymbox}x{\isacharparenright}{\isasymcdot}x\ {\isasymapprox}\ x{\isachardoublequoteclose}\ \ \ \ \ \ \ \ \ \ \ \ \ \ \ \ \ \ \ \ \ \isacommand{using}\isamarkupfalse%
\ A{\isadigit{3}}a\ \isacommand{by}\isamarkupfalse%
\ metis\isanewline
\ \isacommand{have}\isamarkupfalse%
\ \ L{\isadigit{5}}{\isacharcolon}\ \ {\isachardoublequoteopen}{\isasymforall}x{\isachardot}\ {\isasymbox}{\isacharparenleft}{\isacharparenleft}{\isasymbox}{\isasymbox}x{\isacharparenright}{\isasymcdot}x{\isacharparenright}\ {\isasymapprox}\ {\isasymbox}{\isacharparenleft}{\isacharparenleft}{\isasymbox}{\isasymbox}x{\isacharparenright}{\isasymcdot}{\isacharparenleft}{\isasymbox}x{\isacharparenright}{\isacharparenright}{\isachardoublequoteclose}\ \ \ \ \ \isacommand{using}\isamarkupfalse%
\ A{\isadigit{4}}a\ \isacommand{by}\isamarkupfalse%
\ auto\isanewline
\ \isacommand{hence}\isamarkupfalse%
\ L{\isadigit{6}}{\isacharcolon}\ \ {\isachardoublequoteopen}{\isasymforall}x\ {\isachardot}{\isasymbox}{\isacharparenleft}{\isacharparenleft}{\isasymbox}{\isasymbox}x{\isacharparenright}{\isasymcdot}x{\isacharparenright}\ {\isasymapprox}\ {\isasymbox}{\isasymbox}x{\isachardoublequoteclose}\ \ \ \ \ \ \ \ \ \ \ \ \ \ \ \isacommand{using}\isamarkupfalse%
\ A{\isadigit{3}}a\ \isacommand{by}\isamarkupfalse%
\ metis\isanewline
\ \isacommand{hence}\isamarkupfalse%
\ L{\isadigit{7}}{\isacharcolon}\ \ {\isachardoublequoteopen}{\isasymforall}x{\isachardot}\ {\isasymbox}{\isasymbox}{\isacharparenleft}x{\isasymbox}{\isacharparenright}\ {\isasymapprox}\ {\isasymbox}{\isacharparenleft}{\isasymbox}{\isasymbox}{\isacharparenleft}x{\isasymbox}{\isacharparenright}{\isacharparenright}{\isasymcdot}{\isacharparenleft}x{\isasymbox}{\isacharparenright}{\isachardoublequoteclose}\ \ \ \ \ \ \ \isacommand{by}\isamarkupfalse%
\ auto\isanewline
\ \isacommand{hence}\isamarkupfalse%
\ L{\isadigit{8}}{\isacharcolon}\ \ {\isachardoublequoteopen}{\isasymforall}x{\isachardot}\ {\isasymbox}{\isasymbox}{\isacharparenleft}x{\isasymbox}{\isacharparenright}\ {\isasymapprox}\ {\isasymbox}{\isacharparenleft}x{\isasymbox}{\isacharparenright}{\isachardoublequoteclose}\ \ \ \ \ \ \ \ \ \ \ \ \ \ \ \ \ \isacommand{using}\isamarkupfalse%
\ L{\isadigit{4}}\ \isacommand{by}\isamarkupfalse%
\ metis\isanewline
\ \isacommand{hence}\isamarkupfalse%
\ L{\isadigit{9}}{\isacharcolon}\ \ {\isachardoublequoteopen}{\isasymforall}x{\isachardot}\ {\isasymbox}{\isasymbox}{\isacharparenleft}x{\isasymbox}{\isacharparenright}\ {\isasymapprox}\ {\isasymbox}x{\isachardoublequoteclose}\ \ \ \ \ \ \ \ \ \ \ \ \ \ \ \ \ \ \ \ \ \isacommand{using}\isamarkupfalse%
\ A{\isadigit{2}}b\ \isacommand{by}\isamarkupfalse%
\ metis\isanewline
\ \isacommand{hence}\isamarkupfalse%
\ L{\isadigit{1}}{\isadigit{0}}{\isacharcolon}\ {\isachardoublequoteopen}{\isasymforall}x{\isachardot}\ {\isasymbox}{\isasymbox}x\ {\isasymapprox}\ {\isasymbox}x{\isachardoublequoteclose}\ \ \ \ \ \ \ \ \ \ \ \ \ \ \ \ \ \ \ \ \ \ \ \ \isacommand{using}\isamarkupfalse%
\ A{\isadigit{2}}b\ \isacommand{by}\isamarkupfalse%
\ metis\isanewline
\ \isacommand{hence}\isamarkupfalse%
\ L{\isadigit{1}}{\isadigit{1}}{\isacharcolon}\ {\isachardoublequoteopen}{\isasymforall}x{\isachardot}\ {\isasymbox}{\isasymbox}{\isacharparenleft}{\isacharparenleft}{\isasymbox}x{\isacharparenright}{\isasymbox}{\isacharparenright}\ {\isasymapprox}\ {\isasymbox}{\isasymbox}{\isacharparenleft}x{\isasymbox}{\isacharparenright}{\isachardoublequoteclose}\ \ \ \ \ \ \ \ \ \ \ \ \ \isacommand{using}\isamarkupfalse%
\ A{\isadigit{2}}b\ \isacommand{by}\isamarkupfalse%
\ metis\isanewline
\ \isacommand{hence}\isamarkupfalse%
\ L{\isadigit{1}}{\isadigit{2}}{\isacharcolon}\ {\isachardoublequoteopen}{\isasymforall}x{\isachardot}\ {\isasymbox}{\isasymbox}{\isacharparenleft}{\isacharparenleft}{\isasymbox}x{\isacharparenright}{\isasymbox}{\isacharparenright}\ {\isasymapprox}\ {\isasymbox}x{\isachardoublequoteclose}\ \ \ \ \ \ \ \ \ \ \ \ \ \ \ \ \ \ \isacommand{using}\isamarkupfalse%
\ L{\isadigit{9}}\ \isacommand{by}\isamarkupfalse%
\ metis\isanewline
\ \isacommand{have}\isamarkupfalse%
\ \ L{\isadigit{1}}{\isadigit{3}}{\isacharcolon}\ {\isachardoublequoteopen}{\isasymforall}x{\isachardot}\ {\isacharparenleft}{\isasymbox}{\isasymbox}{\isacharparenleft}{\isacharparenleft}{\isasymbox}x{\isacharparenright}{\isasymbox}{\isacharparenright}{\isacharparenright}{\isasymcdot}{\isacharparenleft}{\isacharparenleft}{\isasymbox}x{\isacharparenright}{\isasymbox}{\isacharparenright}\ {\isasymapprox}\ {\isacharparenleft}{\isacharparenleft}{\isasymbox}x{\isacharparenright}{\isasymbox}{\isacharparenright}{\isachardoublequoteclose}\ \ \isacommand{using}\isamarkupfalse%
\ L{\isadigit{4}}\ \isacommand{by}\isamarkupfalse%
\ auto\ \ \ \isanewline
\ \isacommand{hence}\isamarkupfalse%
\ L{\isadigit{1}}{\isadigit{4}}{\isacharcolon}\ {\isachardoublequoteopen}{\isasymforall}x{\isachardot}\ {\isacharparenleft}{\isasymbox}x{\isacharparenright}{\isasymcdot}{\isacharparenleft}{\isacharparenleft}{\isasymbox}x{\isacharparenright}{\isasymbox}{\isacharparenright}\ {\isasymapprox}\ {\isacharparenleft}{\isasymbox}x{\isacharparenright}{\isasymbox}{\isachardoublequoteclose}\ \ \ \ \ \ \ \ \ \ \ \ \isacommand{using}\isamarkupfalse%
\ L{\isadigit{1}}{\isadigit{2}}\ \isacommand{by}\isamarkupfalse%
\ metis\isanewline
\ \isacommand{hence}\isamarkupfalse%
\ L{\isadigit{1}}{\isadigit{5}}{\isacharcolon}\ {\isachardoublequoteopen}{\isasymforall}x{\isachardot}\ {\isacharparenleft}{\isasymbox}x{\isacharparenright}{\isasymbox}\ {\isasymapprox}\ {\isacharparenleft}{\isasymbox}x{\isacharparenright}{\isasymcdot}{\isacharparenleft}{\isacharparenleft}{\isasymbox}x{\isacharparenright}{\isasymbox}{\isacharparenright}{\isachardoublequoteclose}\ \ \ \ \ \ \ \ \ \ \ \ \isacommand{using}\isamarkupfalse%
\ L{\isadigit{1}}{\isadigit{4}}\ \isacommand{by}\isamarkupfalse%
\ auto\isanewline
\ \isacommand{then}\isamarkupfalse%
\ \isacommand{show}\isamarkupfalse%
\ {\isacharquery}thesis\ \isacommand{using}\isamarkupfalse%
\ A{\isadigit{3}}b\ \isacommand{by}\isamarkupfalse%
\ metis\isanewline
\isacommand{qed}\isamarkupfalse%
%
\endisatagproof
{\isafoldproof}%
%
\isadelimproof
\ \ \ \ \ \ \ \ \ \ \ \ \ \ \ \ \ \ \ \ \ \isanewline
%
\endisadelimproof
\isanewline
\isanewline
\isacommand{abbreviation}\isamarkupfalse%
\ DirectedEquality\ {\isacharcolon}{\isacharcolon}\ {\isachardoublequoteopen}e{\isasymRightarrow}e{\isasymRightarrow}bool{\isachardoublequoteclose}\ {\isacharparenleft}\isakeyword{infix}\ {\isachardoublequoteopen}{\isasymgreaterapprox}{\isachardoublequoteclose}\ {\isadigit{6}}{\isadigit{0}}{\isacharparenright}\ \isanewline
\ \isakeyword{where}\ {\isachardoublequoteopen}x\ {\isasymgreaterapprox}\ y\ {\isasymequiv}\ {\isacharparenleft}{\isacharparenleft}E\ x{\isacharparenright}\ \isactrlbold {\isasymrightarrow}\ {\isacharparenleft}E\ y{\isacharparenright}{\isacharparenright}\ \isactrlbold {\isasymand}\ x\ \isactrlbold {\isacharequal}\ y{\isachardoublequoteclose}\ \ \isanewline
\isanewline
\isacommand{lemma}\isamarkupfalse%
\ L{\isadigit{1}}{\isacharunderscore}{\isadigit{1}}{\isadigit{3}}{\isacharcolon}\ {\isachardoublequoteopen}{\isacharparenleft}{\isacharparenleft}{\isasymbox}{\isacharparenleft}x{\isasymcdot}y{\isacharparenright}{\isacharparenright}\ {\isasymapprox}\ {\isacharparenleft}{\isasymbox}{\isacharparenleft}x{\isasymcdot}{\isacharparenleft}{\isasymbox}y{\isacharparenright}{\isacharparenright}{\isacharparenright}{\isacharparenright}\ \isactrlbold {\isasymleftrightarrow}\ {\isacharparenleft}{\isacharparenleft}{\isasymbox}{\isacharparenleft}x{\isasymcdot}y{\isacharparenright}{\isacharparenright}\ {\isasymgreaterapprox}\ {\isasymbox}x{\isacharparenright}{\isachardoublequoteclose}\ \isanewline
%
\isadelimproof
%
\endisadelimproof
%
\isatagproof
\isacommand{by}\isamarkupfalse%
\ {\isacharparenleft}metis\ A{\isadigit{1}}\ A{\isadigit{2}}a\ A{\isadigit{3}}a{\isacharparenright}%
\endisatagproof
{\isafoldproof}%
%
\isadelimproof
\isanewline
%
\endisadelimproof
\isanewline
\isanewline
\isacommand{lemma}\isamarkupfalse%
\ {\isachardoublequoteopen}{\isacharparenleft}\isactrlbold {\isasymexists}x{\isachardot}\ e\ {\isasymapprox}\ {\isacharparenleft}{\isasymbox}x{\isacharparenright}{\isacharparenright}\ \isactrlbold {\isasymleftrightarrow}\ {\isacharparenleft}\isactrlbold {\isasymexists}x{\isachardot}\ e\ {\isasymapprox}\ {\isacharparenleft}x{\isasymbox}{\isacharparenright}{\isacharparenright}{\isachardoublequoteclose}%
\isadelimproof
\ %
\endisadelimproof
%
\isatagproof
\isacommand{by}\isamarkupfalse%
\ {\isacharparenleft}metis\ A{\isadigit{1}}\ A{\isadigit{2}}b\ A{\isadigit{3}}b{\isacharparenright}%
\endisatagproof
{\isafoldproof}%
%
\isadelimproof
%
\endisadelimproof
\isanewline
\isacommand{lemma}\isamarkupfalse%
\ {\isachardoublequoteopen}{\isacharparenleft}\isactrlbold {\isasymexists}x{\isachardot}\ e\ {\isasymapprox}\ {\isacharparenleft}x{\isasymbox}{\isacharparenright}{\isacharparenright}\ \isactrlbold {\isasymleftrightarrow}\ e\ {\isasymapprox}\ {\isacharparenleft}{\isasymbox}e{\isacharparenright}{\isachardoublequoteclose}%
\isadelimproof
\ \ \ \ \ \ \ %
\endisadelimproof
%
\isatagproof
\isacommand{by}\isamarkupfalse%
\ {\isacharparenleft}metis\ A{\isadigit{1}}\ A{\isadigit{2}}b\ A{\isadigit{3}}a\ A{\isadigit{3}}b{\isacharparenright}%
\endisatagproof
{\isafoldproof}%
%
\isadelimproof
%
\endisadelimproof
\isanewline
\isacommand{lemma}\isamarkupfalse%
\ {\isachardoublequoteopen}e\ {\isasymapprox}\ {\isacharparenleft}{\isasymbox}e{\isacharparenright}\ \isactrlbold {\isasymleftrightarrow}\ e\ {\isasymapprox}\ {\isacharparenleft}e{\isasymbox}{\isacharparenright}{\isachardoublequoteclose}%
\isadelimproof
\ \ \ \ \ \ \ \ \ \ \ \ \ %
\endisadelimproof
%
\isatagproof
\isacommand{by}\isamarkupfalse%
\ {\isacharparenleft}metis\ A{\isadigit{1}}\ A{\isadigit{2}}b\ A{\isadigit{3}}a\ A{\isadigit{3}}b\ A{\isadigit{4}}a{\isacharparenright}%
\endisatagproof
{\isafoldproof}%
%
\isadelimproof
%
\endisadelimproof
\isanewline
\isacommand{lemma}\isamarkupfalse%
\ {\isachardoublequoteopen}e\ {\isasymapprox}\ {\isacharparenleft}e{\isasymbox}{\isacharparenright}\ \isactrlbold {\isasymleftrightarrow}\ {\isacharparenleft}\isactrlbold {\isasymforall}x{\isachardot}\ e{\isasymcdot}x\ {\isasymgreaterapprox}\ x{\isacharparenright}{\isachardoublequoteclose}%
\isadelimproof
\ \ \ \ \ \ \ \ \ %
\endisadelimproof
%
\isatagproof
\isacommand{by}\isamarkupfalse%
\ {\isacharparenleft}metis\ A{\isadigit{1}}\ A{\isadigit{2}}b\ A{\isadigit{3}}a\ A{\isadigit{3}}b\ A{\isadigit{4}}a{\isacharparenright}%
\endisatagproof
{\isafoldproof}%
%
\isadelimproof
%
\endisadelimproof
\ \isanewline
\isacommand{lemma}\isamarkupfalse%
\ {\isachardoublequoteopen}{\isacharparenleft}\isactrlbold {\isasymforall}x{\isachardot}\ e{\isasymcdot}x\ {\isasymgreaterapprox}\ x{\isacharparenright}\ \isactrlbold {\isasymleftrightarrow}\ {\isacharparenleft}\isactrlbold {\isasymforall}x{\isachardot}\ x{\isasymcdot}e\ {\isasymgreaterapprox}\ x{\isacharparenright}{\isachardoublequoteclose}%
\isadelimproof
\ \ \ \ \ %
\endisadelimproof
%
\isatagproof
\isacommand{by}\isamarkupfalse%
\ {\isacharparenleft}metis\ A{\isadigit{1}}\ A{\isadigit{2}}b\ A{\isadigit{3}}a\ A{\isadigit{3}}b{\isacharparenright}%
\endisatagproof
{\isafoldproof}%
%
\isadelimproof
%
\endisadelimproof
\isanewline
\isanewline
\isanewline
\isacommand{abbreviation}\isamarkupfalse%
\ IdentityMorphism\ {\isacharcolon}{\isacharcolon}\ {\isachardoublequoteopen}e{\isasymRightarrow}bool{\isachardoublequoteclose}\ {\isacharparenleft}{\isachardoublequoteopen}IdM{\isacharunderscore}{\isachardoublequoteclose}\ {\isacharbrackleft}{\isadigit{1}}{\isadigit{0}}{\isadigit{0}}{\isacharbrackright}{\isadigit{6}}{\isadigit{0}}{\isacharparenright}\ \isakeyword{where}\ {\isachardoublequoteopen}IdM\ x\ {\isasymequiv}\ x\ {\isasymapprox}\ {\isacharparenleft}{\isasymbox}x{\isacharparenright}{\isachardoublequoteclose}\isanewline
\isanewline
\isacommand{lemma}\isamarkupfalse%
\ {\isachardoublequoteopen}{\isacharparenleft}IdM\ e\ \isactrlbold {\isasymleftrightarrow}\ {\isacharparenleft}\isactrlbold {\isasymexists}x{\isachardot}\ e\ {\isasymapprox}\ {\isacharparenleft}{\isasymbox}x{\isacharparenright}{\isacharparenright}{\isacharparenright}\ \isactrlbold {\isasymand}\isanewline
\ \ \ \ \ \ \ {\isacharparenleft}IdM\ e\ \isactrlbold {\isasymleftrightarrow}\ {\isacharparenleft}\isactrlbold {\isasymexists}x{\isachardot}\ e\ {\isasymapprox}\ {\isacharparenleft}x{\isasymbox}{\isacharparenright}{\isacharparenright}{\isacharparenright}\ \isactrlbold {\isasymand}\ \isanewline
\ \ \ \ \ \ \ {\isacharparenleft}IdM\ e\ \isactrlbold {\isasymleftrightarrow}\ e\ {\isasymapprox}\ {\isacharparenleft}{\isasymbox}e{\isacharparenright}{\isacharparenright}\ \isactrlbold {\isasymand}\isanewline
\ \ \ \ \ \ \ {\isacharparenleft}IdM\ e\ \isactrlbold {\isasymleftrightarrow}\ e\ {\isasymapprox}\ {\isacharparenleft}e{\isasymbox}{\isacharparenright}{\isacharparenright}\ \isactrlbold {\isasymand}\isanewline
\ \ \ \ \ \ \ {\isacharparenleft}IdM\ e\ \isactrlbold {\isasymleftrightarrow}\ {\isacharparenleft}\isactrlbold {\isasymforall}x{\isachardot}\ e{\isasymcdot}x\ {\isasymgreaterapprox}\ x{\isacharparenright}{\isacharparenright}\ \isactrlbold {\isasymand}\isanewline
\ \ \ \ \ \ \ {\isacharparenleft}IdM\ e\ \isactrlbold {\isasymleftrightarrow}\ {\isacharparenleft}\isactrlbold {\isasymforall}x{\isachardot}\ x{\isasymcdot}e\ {\isasymgreaterapprox}\ x{\isacharparenright}{\isacharparenright}{\isachardoublequoteclose}\isanewline
%
\isadelimproof
\ %
\endisadelimproof
%
\isatagproof
\isacommand{by}\isamarkupfalse%
\ {\isacharparenleft}smt\ A{\isadigit{1}}\ A{\isadigit{2}}a\ A{\isadigit{3}}a\ A{\isadigit{3}}b{\isacharparenright}%
\endisatagproof
{\isafoldproof}%
%
\isadelimproof
\isanewline
%
\endisadelimproof
%
\isadelimtheory
%
\endisadelimtheory
%
\isatagtheory
\isacommand{end}\isamarkupfalse%
%
\endisatagtheory
{\isafoldtheory}%
%
\isadelimtheory
%
\endisadelimtheory
\end{isabellebody}%
%%% Local Variables:
%%% mode: latex
%%% TeX-master: "root"
%%% End:


%%% Local Variables:
%%% mode: latex
%%% TeX-master: "root"
%%% End:


% optional bibliography
\bibliographystyle{abbrv}
\bibliography{root}

\end{document}

%%% Local Variables:
%%% mode: latex
%%% TeX-master: t
%%% End:
