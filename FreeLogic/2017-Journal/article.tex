%%%%%%%%%%%%%%%%%%%%%%% file template.tex %%%%%%%%%%%%%%%%%%%%%%%%%
%
% This is a general template file for the LaTeX package SVJour3
% for Springer journals.          Springer Heidelberg 2010/09/16
%
% Copy it to a new file with a new name and use it as the basis
% for your article. Delete % signs as needed.
%
% This template includes a few options for different layouts and
% content for various journals. Please consult a previous issue of
% your journal as needed.
%
%%%%%%%%%%%%%%%%%%%%%%%%%%%%%%%%%%%%%%%%%%%%%%%%%%%%%%%%%%%%%%%%%%%
%
% First comes an example EPS file -- just ignore it and
% proceed on the \documentclass line
% your LaTeX will extract the file if required
\begin{filecontents*}{example.eps}
%!PS-Adobe-3.0 EPSF-3.0
%%BoundingBox: 19 19 221 221
%%CreationDate: Mon Sep 29 1997
%%Creator: programmed by hand (JK)
%%EndComments
gsave
newpath
  20 20 moveto
  20 220 lineto
  220 220 lineto
  220 20 lineto
closepath
2 setlinewidth
gsave
  .4 setgray fill
grestore
stroke
grestore
\end{filecontents*}
%
\RequirePackage{fix-cm}
%
%\documentclass{svjour3}                     % onecolumn (standard format)
%\documentclass[smallcondensed]{svjour3}     % onecolumn (ditto)
\documentclass[smallextended]{svjour3}       % onecolumn (second format)
%\documentclass[twocolumn]{svjour3}          % twocolumn
%
\smartqed  % flush right qed marks, e.g. at end of proof
%
\usepackage{graphicx}
%
% \usepackage{mathptmx}      % use Times fonts if available on your TeX system
%
% insert here the call for the packages your document requires
%\usepackage{latexsym}
% etc.
%
% please place your own definitions here and don't use \def but
% \newcommand{}{}
%
% Insert the name of "your journal" with
% \journalname{myjournal}
%
\begin{document}

\title{Exploring Axiom Systems for Category Theory%\thanks{Grants or other notes
%about the article that should go on the front page should be
%placed here. General acknowledgments should be placed at the end of the article.}
}
\subtitle{Utilizing a 
 Novel Approach to Automate Free Logic in HOL}

%\titlerunning{Short form of title}        % if too long for running head

\author{Christoph Benzmüller       \and
      Dana Scott 
}

%\authorrunning{Short form of author list} % if too long for running head

\institute{F. Author \at
              first address \\
              Tel.: +123-45-678910\\
              Fax: +123-45-678910\\
              \email{fauthor@example.com}           %  \\
%             \emph{Present address:} of F. Author  %  if needed
           \and
           S. Author \at
              second address
}

\date{Received: date / Accepted: date}
% The correct dates will be entered by the editor


\maketitle

\begin{abstract}
  % Partiality and undefinedness are prominent challenges in various
  % areas of mathematics and computer science.  Unfortunately, however,
  % modern proof assistant system based on traditional classical or
  % intuitionistic logics provide rather inadequate support for these
  % challenge concepts.  Free logic offers a theoretically appealing
  % solution, but is has been considered as rather unsuited towards
  % practical utilisation.

  % The two main contributions of this article are: (i) A shallow
  % semantical embedding of free logic in classical higher-order logic
  % is presented; this embedding enables the application of higher-order
  % interactive and automated theorem provers (and their integrated
  % subprovers) for the formalisation and verification of free logic
  % theories.  (ii) This approach to automate free logic is exemplarily
  % employed in a selected domain of mathematics: Starting from a
  % generalization of the standard axioms for a monoid we present a
  % stepwise development of various, mutually equivalent foundational
  % axiom systems for category theory. As a side-effect of this work
  % some (minor) issue in a prominent category theory textbook have been
  % revealed.

  A shallow semantical embedding of free logic in classical
  higher-order logic is presented, which enables the off-the-shelf
  application of higher-order interactive and automated theorem
  provers (and their integrated subprovers) for the formalisation and
  verification of free logic theories.  
  Subsequently, this approach is exemplarily employed in a selected domain of
  mathematics: starting from a generalization of the standard axioms
  for a monoid we present a stepwise development of various, mutually
  equivalent foundational axiom systems for category theory. As a
  side-effect of this work some (minor) issue in a prominent category
  theory textbook has been revealed.


\keywords{Free Logic \and Classical Higher-Order Logic \and Category
  Theory \and Interactive and Automated Theorem Proving }
% \PACS{PACS code1 \and PACS code2 \and more}
% \subclass{MSC code1 \and MSC code2 \and more}
\end{abstract}

\section{Introduction}
\label{intro}
Partiality and undefinedness are prominent challenges in various areas
of mathematics and computer science.  Unfortunately, however, modern
proof assistant systems and automated theorem provers based on
traditional classical or intuitionistic logics provide rather
inadequate support for these challenge concepts.  Free logic offers a
theoretically appealing solution, but is has been considered as rather
unsuited towards practical utilisation.


In this article we show how free logic can be
``implemented'' in any theorem proving system for classical
higher-order logic (HOL) \cite{B5}. The proposed solution employs a
semantic embedding of free (or inclusive logic) in HOL. We present an
exemplary implementation of this idea in the mathematical proof
assistant Isabelle/HOL \cite{NPW02}. Various state-of-the-art
first-order and higher-order automated theorem provers and model
finders are integrated (modulo suitable logic translations) with
Isabelle via the Sledgehammer tool \cite{Sledgehammer}, so that our
solution can be utilized, via Isabelle as foreground system, with a
whole range of other background reasoners. As a result we obtain an
elegant and powerful implementation of an interactive and automated
theorem proving (and model finding) system for free logic.


  The two main contributions of this article are: (i) A shallow
  semantical embedding of free logic in classical higher-order logic
  is presented; this embedding enables the application of higher-order
  interactive and automated theorem provers (and their integrated
  subprovers) for the formalisation and verification of free logic
  theories.  (ii) This approach to automate free logic is exemplarily
  employed in a selected domain of mathematics: Starting from a
  generalization of the standard axioms for a monoid we present a
  stepwise development of various, mutually equivalent foundational
  axiom systems for category theory. As a side-effect of this work
  some (minor) issue in a prominent category theory textbook have been
  revealed.



Partiality and undefinedness are core concepts in various areas of
mathematics.  Modern mathematical proof assistants and theorem proving
systems are often based on traditional classical or intuitionistic
logics and provide rather inadequate support for these challenge
concepts.  Free logic @{cite "sep-logic-free,scott67:exist"}, in
contrast, offers a theoretically and practically appealing
solution. Unfortunately, however, we are not aware of any implemented
and available theorem proving system for free logic.
 
In this extended abstract we show how free logic can be
``implemented'' in any theorem proving system for classical
higher-order logic (HOL) @{cite "B5"}. The proposed solution employs a
semantic embedding of free (or inclusive logic) in HOL. We present an
exemplary implementation of this idea in the mathematical proof
assistant Isabelle/HOL @{cite "NPW02"}. Various state-of-the-art
first-order and higher-order automated theorem provers and model
finders are integrated (modulo suitable logic translations) with
Isabelle via the Sledgehammer tool @{cite "Sledgehammer"}, so that our
solution can be utilized, via Isabelle as foreground system, with a
whole range of other background reasoners. As a result we obtain an
elegant and powerful implementation of an interactive and automated
theorem proving (and model finding) system for free logic.
  
To demonstrate the practical relevance of our new system, we present a
stepwise development of axiom systems for category theory by
generalizing the standard axioms for a monoid to a partial composition
operation. Our purpose is not to make or claim any contribution to
category theory but rather to show how formalizations involving the
kind of logic required (free logic) can be validated within modern
proof assistants.

A total of eight different axiom systems is studied. The systems I-VI are shown to 
be equivalent. The axiom system VII slightly modifies axiom system VI to obtain (modulo 
notational transformation) the set of axioms as proposed by  Freyd and Scedrov in their textbook
 ``Categories, Allegories'' @{cite "FreydScedrov90"}, published in 1990; 
see also Subsection \ref{subsec:FreydNotation} where we present their original system.
While the axiom systems I-VI are shown to be  consistent, a constricted inconsistency result is 
obtained for system VII (when encoded in free logic where free variables range over all 
objects): We can prove @{text "(∃x. ❙¬(E x)) ❙→ False"}, where @{text "E"} is the existence predicate. Read this as: If there 
are undefined objects, e.g. the value of an undefined composition @{text "x⋅y"}, then we have falsity.
By contraposition, all objects (and thus all compositions) must exist. But when we assume the latter,
then the axiom system VII essentially reduces categories to monoids.
We note that axiom system V, which avoids this problem, corresponds to a set of axioms proposed 
by Scott @{cite "Scott79"} in the 1970s. The problem can also be avoided by restricting the variables 
in axiom system VII to range only over existing objects and by postulating strictness conditions. 
This gives us axiom system VIII.

Our exploration has been significantly supported by series of experiments in which automated reasoning tools 
have been called from within the proof assistant Isabelle/HOL @{cite "Isabelle"} via the Sledgehammer 
tool @{cite "Sledgehammer"}. Moreover, we have obtained very useful feedback at various stages 
from the model finder Nitpick @{cite "Nitpick"} saving us from making several mistakes.

At the conceptual level this paper exemplifies a new style of explorative mathematics which rests 
on a significant amount of human-machine interaction with integrated interactive-auto\-ma\-ted 
theorem proving technology. The experiments we have conducted are such that the required 
reasoning is often too tedious and time-consuming for humans to be carried out repeatedly with 
highest level of precision. It is here where cycles of formalization and experimentation efforts in 
Isabelle/HOL provided  significant support. Moreover, the technical inconsistency issue for
axiom system VII was discovered by automated theorem provers, which further emphasises the added 
value of automated theorem proving in this area. 

To enable our experiments we have exploited an embedding of free logic @{cite "Scott67"} 
in classical higher-order logic, which we have recently presented in a related paper @{cite "C57"}.


We also want to emphasize that this paper has been written entirely within the Isabelle 
framework by utilizing the Isabelle ``build'' tool; cf. @{cite "IsabelleManual2016"}, Section~2. 
It is thus an example of a formally verified mathematical document, where the PDF document as 
presented here has been generated directly from the verified source files mentioned above.
We also note that once the proofs have been mechanically checked, they are generally easy 
to find by hand using paper and pencil.


\section{Section title}
\label{sec:1}
Text with citations \cite{RefB} and \cite{RefJ}.
\subsection{Subsection title}
\label{sec:2}
as required. Don't forget to give each section
and subsection a unique label (see Sect.~\ref{sec:1}).
\paragraph{Paragraph headings} Use paragraph headings as needed.
\begin{equation}
a^2+b^2=c^2
\end{equation}

% For one-column wide figures use
\begin{figure}
% Use the relevant command to insert your figure file.
% For example, with the graphicx package use
  \includegraphics{example.eps}
% figure caption is below the figure
\caption{Please write your figure caption here}
\label{fig:1}       % Give a unique label
\end{figure}
%
% For two-column wide figures use
\begin{figure*}
% Use the relevant command to insert your figure file.
% For example, with the graphicx package use
  \includegraphics[width=0.75\textwidth]{example.eps}
% figure caption is below the figure
\caption{Please write your figure caption here}
\label{fig:2}       % Give a unique label
\end{figure*}
%
% For tables use
\begin{table}
% table caption is above the table
\caption{Please write your table caption here}
\label{tab:1}       % Give a unique label
% For LaTeX tables use
\begin{tabular}{lll}
\hline\noalign{\smallskip}
first & second & third  \\
\noalign{\smallskip}\hline\noalign{\smallskip}
number & number & number \\
number & number & number \\
\noalign{\smallskip}\hline
\end{tabular}
\end{table}


%\begin{acknowledgements}
%If you'd like to thank anyone, place your comments here
%and remove the percent signs.
%\end{acknowledgements}

% BibTeX users please use one of
%\bibliographystyle{spbasic}      % basic style, author-year citations
\bibliographystyle{spmpsci}      % mathematics and physical sciences
%\bibliographystyle{spphys}       % APS-like style for physics
\bibliography{bibliography}   % name your BibTeX data base

% Non-BibTeX users please use
\begin{thebibliography}{}
%
% and use \bibitem to create references. Consult the Instructions
% for authors for reference list style.
%
\bibitem{RefJ}
% Format for Journal Reference
Author, Article title, Journal, Volume, page numbers (year)
% Format for books
\bibitem{RefB}
Author, Book title, page numbers. Publisher, place (year)
% etc
\end{thebibliography}

\end{document}
% end of file template.tex

